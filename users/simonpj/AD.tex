\documentclass{article}

\usepackage{color}
\usepackage{listings}
\lstset{
  mathescape=true
}
\usepackage{amssymb}

\renewcommand{\to}{\rightarrow}    % ->
\newcommand{\linto}{\multimap}   % -o
\newcommand{\grad}[1]{\nabla\! f}  % Full Jacobian
\newcommand{\fwdD}[1]{f'}  % Forward derivative, f'
\newcommand{\revD}[1]{f`}  % Reverse derivative, f`


\begin{document}


\section{Basic setup}

$$
\begin{array}{lrcl}
  \mbox{Original function}   & f        & : & S \to T \\
  \mbox{Full Jacobian}       & \grad{f} & : & S \to (S \linto T) \\
  \mbox{Forward derivative}  & \fwdD{f} & : & S \to S \to T \\
  \mbox{Reverse derivative}  & \revD{f} & : & S \to T \to S \\
\end{array}
$$

\begin{lstlisting}
  f x = x
  g y = let v = 3 $\circ$ 7
            w = 4
            t = 7
        in wueble
\end{lstlisting}
\end{document}


"The full Jacobian"
If           f : S→T
then  ∇f :S→(S⊸T)

Forward mode:
f^′:S→S→T
f^′ x dx=∇f(x)⋆dx

Backward mode
f` :S→T→S
f` x dr=(∇f(x))^⊤⋆dr

The API of linear maps S⊸T
Apply:      (⋆)  :(s⊸t)→s→t
Compose:(∘)  : (s⊸t)→(r⊸s)→(r⊸t)
Transpose:  ∎^⊤  :  (s⊸t)→(t⊸s)
Pairing:       ( ,)  :Field s⇒ (s⊸t_1 )→(s⊸t_2 )→(s⊸(t_1, t_2 ))    -- vertical concat [J1; J2]
Joining:      (⋈)  :Field t⇒ (s_1⊸t)→(s_2⊸t)→((s_1,s_2)⊸t))   -- horizontal concat [J1 | J2]
Adding:      (+)  :Field t⇒(s⊸t)→ (s⊸t)→ (s⊸t)
Zero:  0 :Field t⇒ s⊸t
Unit:  1 :s⊸s
Scalar:   Scalar :Field s⇒s→(s⊸s)

Conal says
	• (,) is what he calls "fork", and Control.Arrow calls (&&&).Moreover, if you think of the realisation of linear maps as matrices, then (,) is vertical juxtaposition
	
	• (⋈)  is what he calls "join", and Control.Arrow calls (|||).Moreoer, in the matrix interpretation, (⋈)  is horizontal juxatposition.

I am not sure if "Field" is the right term here, but let's write down the operations we need.  Tom thinks we do not need (*):
	class Field a where
		  zero∷a
		  one∷a
		  (+)∷a→a→a
		  (∗)∷a→a→a
		  fromInteger∷Integer →a
		

Laws of linear maps
Scalar(a)⋆x=a∗x
(m_1 〖∘m〗_2 )⋆x= m_1⋆ (m_2⋆x)
(m_1, m_2 )⋆x=(m_1⋆x, m_2⋆x)
(m_1⋈m_2 )⋆p=〖(m〗_1⋆fst(p))+(m_2⋆snd(p))
(m_1 〖+ m〗_2 )⋆x=〖(m〗_1⋆x)+(m_2⋆x)

0⋆x=0
1⋆x=x

0∘m=m∘0=0
1∘m=m∘1=m

m ∘ (n_1⋈ n_2 )=(m∘n_1 )⋈(m∘n_2 )
(m_1⋈m_2 )  ∘ (n_1,n_2 )=(m_1 〖∘n〗_1 )+(m_2 〖∘n〗_2 )
	This one seems to come up in practice, and led us to add (+) on linear maps
	
Scalar(x)+Scalar(y)=Scalar(x+y)
Scalar(x)∘Scalar(y)=Scalar( x∗y )

(0,0)=0

0^⊤=0
1^⊤=1
(Scalar(f))^⊤=Scalar(f)
(m_1 〖,m〗_2 )^⊤= 〖m_1〗^⊤⋈〖m_2〗^⊤
(m_1 〖⋈m〗_2 )^⊤=(m_1^⊤,m_2^⊤ )
(m_1 〖∘m〗_2 )^⊤= m_2^⊤  ∘m_1^⊤
〖(m^⊤)〗^⊤=m


Primitives
∇ +  :(Float,Float)→((Float,Float)  ⊸Float)
∇+(p,q)=1⋈1

∇∗  :(Float,Float)→((Float,Float)  ⊸Float)
∇∗(p,q)=Scalar(q)⋈Scalar(p)

∇"fst" :(S,T)→((S,T)⊸S)
∇fst(p,q)= 1⋈0

Syntax of the language

f, g, h∷=names of functions
x,y,x∷=names of variables
∇f, ∇x∷=names of gradified versions of f,x
k∷=literal constants

pgm∷=def_1… def_n
def∷=fun f(x)=e
e∷=k        | x         | f(e)  
         | (e_1 〖, e〗_2)
         | let x=e_1  in e_2


Computing grad

Definitions
∇⟦fun f(x)=e⟧=
      fun ∇f(x)=let ∇x=1 in ∇⟦e⟧

Expressions
Invariant: If e :T, and there exists S that for each free variable p:P in e, we have ∇p :S ⊸P
 then ∇⟦e⟧  :S⊸T, 

∇⟦k⟧=0
∇⟦x⟧=∇x
∇⟦(e_1 〖, e〗_2)⟧=(∇⟦e_1 ⟧ 〖,∇ ⟦e_2 ⟧〗_ )
∇⟦f(e)⟧=∇f(e)∘ ∇⟦e⟧
∇⟦let y=e_1  in e_2 ⟧=  let y=e_(1 )  in                                          let ∇y=∇⟦e_1 ⟧  in
                                          ∇⟦e_2 ⟧
